\section{Poly\-CDI\_\-simulation\_\-example.c File Reference}
\label{PolyCDI__simulation__example_8c}\index{PolyCDI_simulation_example.c@{PolyCDI\_\-simulation\_\-example.c}}
{\tt \#include $<$iostream$>$}\par
{\tt \#include $<$cmath$>$}\par
{\tt \#include $<$cstring$>$}\par
{\tt \#include $<$cstdlib$>$}\par
{\tt \#include $<$io.h$>$}\par
{\tt \#include $<$Complex\_\-2D.h$>$}\par
{\tt \#include $<$Double\_\-2D.h$>$}\par
{\tt \#include $<$Poly\-CDI.h$>$}\par
{\tt \#include $<$sstream$>$}\par
\subsection*{Functions}
\begin{CompactItemize}
\item 
int \bf{main} (void)\label{PolyCDI__simulation__example_8c_5ad1c26f00c2399d3a5c7850100212d0}

\end{CompactItemize}


\subsection{Detailed Description}
{\em Polyc\-CDI\_\-simulation\_\-example.c\/} The object from the Planar\-CDI\_\-example (in fact I used the reconstructed image as the object) is used to simulate a diffraction pattern for polychromatic CDI. A beam is simulated according to the properties of the beam of thee Australain Synchrotron. The spectrum for this was generated using SPECTRA (\tt{http://radiant.harima.riken.go.jp/spectra/}) and saved as a text file. The diffraction pattern is thresholded to make it more realistic. 

Definition in file \bf{Poly\-CDI\_\-simulation\_\-example.c}.