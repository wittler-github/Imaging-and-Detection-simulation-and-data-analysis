\section{dbin2tiff.c File Reference}
\label{dbin2tiff_8c}\index{dbin2tiff.c@{dbin2tiff.c}}
{\tt \#include $<$iostream$>$}\par
{\tt \#include $<$stdlib.h$>$}\par
{\tt \#include $<$Double\_\-2D.h$>$}\par
{\tt \#include $<$io.h$>$}\par
\subsection*{Functions}
\begin{CompactItemize}
\item 
int \bf{main} (int argc, char $\ast$argv[$\,$])\label{dbin2tiff_8c_28052c36c3b61c6c0eaa18f5d226118f}

\end{CompactItemize}


\subsection{Detailed Description}
{\em dbin2tiff.exe\/} - Convert a binary file (2D double / 64 bit, format) to a tiff (grey-scale, 16 bit file).

\begin{Desc}
\item[Usage: dbin2tiff.exe $<$input dbin file$>$ $<$output tiff file$>$]$<$pixels in x$>$ $<$pixels in y$>$\end{Desc}
\begin{Desc}
\item[]\end{Desc}
\begin{Desc}
\item[Example:]

\footnotesize\begin{verbatim}dbin2tiff.exe my_reconstruction.dbin my_reconstruction.tiff 1024 1024 \end{verbatim}
\normalsize
 \end{Desc}


Definition in file \bf{dbin2tiff.c}.